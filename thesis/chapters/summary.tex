\summary
Celem mojej pracy było zbadanie efektywności SMT solverów w rozwiązywaniu klasycznych problemów NP-trudnych. W tym celu zakodowałam osiem wybranych zagadnień z tej kategorii, w tym siedem z teorii grafów i jeden z optymalizacji kombinatorycznej, i przeprowadziłam serię eksperymentów mających na celu zrozumienie, jak solvery Z3, Yices i cvc5 radzą sobie z tymi problemami obliczeniowymi.

Uważam, że udało mi się zrealizować ten cel, gdyż analiza wyników eksperymentów pozwoliła mi na dokładne zrozumienie wydajności poszczególnych solverów w różnych scenariuszach. Dzięki temu mam lepsze rozeznanie w tym, jakie rodzaje problemów mogą być rozwiązywane efektywnie przez poszczególne narzędzia, co może być przydatne w praktycznych zastosowaniach oraz w dalszych badaniach naukowych.

Wyzwaniem naukowym jest rozwój kodowania większej liczby problemów NP-trudnych z różnych dziedzin oraz przeprowadzenie eksperymentów na rozleglejszych zestawach danych wejściowych. Rozszerzenie zbioru problemów pozwoliłoby na bardziej wszechstronne zbadanie możliwości i ograniczeń poszczególnych solverów SMT w kontekście problemów NP-trudnych. Dodatkowo, większa różnorodność problemów pozwoliłaby na lepsze zrozumienie zachowania solverów w różnych scenariuszach, co może prowadzić do lepszych praktycznych zastosowań tych narzędzi w rozwiązywaniu rzeczywistych problemów.