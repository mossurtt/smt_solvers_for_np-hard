\chapter{Eksperymenty i analiza wyników}

W procesie generowania danych wejściowych, takich jak grafy, przyjęto zastosowanie biblioteki igraph, która stanowi potężne narzędzie w dziedzinie analizy grafów. Biblioteka ta oferuje wszechstronne możliwości tworzenia, manipulowania oraz analizy grafów, co czyni ją popularnym wyborem wśród badaczy i praktyków zajmujących się analizą sieci oraz grafów. Zaawansowane funkcje igraph umożliwiają generowanie różnorodnych typów grafów, w tym grafów skierowanych i nieskierowanych, o różnych rozmiarach i strukturach. Dodatkowo, igraph dostarcza mechanizmy do manipulacji wierzchołkami oraz krawędziami grafu, co pozwala na elastyczne dostosowanie danych wejściowych do potrzeb konkretnego badania czy eksperymentu. 


\section{Przeprowadzenie eksperymentów i zbieranie danych}

\section{Analiza zebranych danych i porównanie efektywności solverów}

\section{Identyfikacja czynników wpływających na efektywność}

