\chapter{Teoretyczne podstawy Satisfiability Modulo Theories}

\section{Wprowadzenie do SMT}

\subsection{Problem spełnialnośći}
Problem spełnialności formuł boolowskich (SAT) jest ważnym problemem algorytmicznym w teorii złożoności obliczeniowej.

Obiektem problemu SAT jest formuła boolowska składająca się jedynie z nazw zmiennych, nawiasów i operacji 
{\wedge} (AND), {\vee} (OR) oraz {\neg} (NOT). Problem polega na tym, czy możliwe jest przypisanie wartości false i true do wszystkich zmiennych występujących w formule, tak aby formuła stała się prawdziwa.

Zgodnie z twierdzeniem Cooka, udowodnionym przez Stephena Cooka w 1971 roku, problem SAT dla formuł boolowskich zapisanych w koniunkcyjnej postaci normalnej jest NP-zupełny. Wymóg zapisu w postaci koniunkcyjnej jest ważny, ponieważ, na przykład, dla formuł reprezentowanych w dysjunkcyjnej postaci normalnej, problem SAT jest trywialnie rozwiązywany w czasie liniowym względem rozmiaru formuły.

\subsection{Satisfiability Modulo Theories}
Satisfiability Modulo Theories (SMT) to dziedzina informatyki teoretycznej, która łączy w sobie problem spełnialności logicznej (SAT) z różnymi teoriami matematycznymi \cite{MouraB11}. 
W kontekście SMT, dany jest zestaw ograniczeń logicznych wyrażonych za pomocą formuł logiki pierwszego rzędu oraz dodatkowe ograniczenia wynikające z konkretnych teorii matematycznych, takich jak teoria liczb całkowitych, teoria równań różniczkowych, czy teoria tablic.
Problematyka SMT polega na stwierdzeniu, czy istnieją wartości zmiennych spełniające zarówno ograniczenia logiczne, jak i dodatkowe ograniczenia wynikające z wybranej teorii matematycznej. W przypadku pozytywnej odpowiedzi, rozwiązaniem problemu jest znaczenie konkretnych wartości zmiennych, które spełniają wszystkie warunki.
	
\section{Historia i rozwój SMT solverów}
Pierwsze próby rozwiązywania problemów SMT miały na celu przekształcenie ich w formuły SAT (na przykład 32-bitowe zmienne zostały zakodowane przez 32 zmienne boolowskie, a odpowiadające im operacje na słowach zostały zakodowane jako niskopoziomowe operacje bitowe) i rozwiązanie formuły za pomocą SAT solvera. Podejście to ma swoje zalety - pozwala na wykorzystanie istniejących solverów SAT bez zmian (As-Is), a także na wykorzystanie wprowadzonych w nich optymalizacji. Z drugiej strony, utrata wysokopoziomowej semantyki leżącej u podstaw teorii oznacza, że solver SAT musi podjąć znaczne działania, aby wywnioskować "oczywiste" fakty (takie jak $x + y = y + x$ dla dodawania). Pomysł ten doprowadził do powstania wyspecjalizowanych solverów SMT, które integrują konwencjonalne dowody boolowskie w stylu algorytmu DPLL z solverami specyficznymi dla teorii ("T solvery"), które działają z dysjunkcjami i koniunkcjami predykatów z danej teorii. 
Satisfiability Modulo Theories(SMT) uogólniają teorię spełnialności boole'owskiej (SAT) poprzez dodanie rozumowania równościowego, arytmetyki, bit-wektorów o stałym rozmiarze, tablic, kwantyfikatorów i innych przydatnych teorii pierwszego rzędu.
SMT solver jest narzędziem do decydowania o spełnialności formuł w tych teoriach. 
%Solvery SMT umożliwiają aplikacje, takie jak rozszerzone sprawdzanie statyczne, abstrakcja predykatów, generowanie przypadków testowych i ograniczone sprawdzanie modelowe w nieskończonych dziedzinach.

\section{Podstawowe teorie logiczne}
% Omówienie kluczowych teorii logicznych, które są obsługiwane przez SMT-solvery. W tym miejscu warto skoncentrować się na teoriach istotnych dla klasycznych problemów NP-trudnych.

\section{Zastosowanie SMT w praktyce}
SMT solvery są powszechnie uznawane za niezbędne mechanizmy rozumowania dla różnych obszarów zastosowań, w tym weryfikacji oprogramowania i sprzętu, sprawdzania modelowego, typowania, analizy statycznej, bezpieczeństwa, automatycznego generowania przypadków testowych, syntezy, planowania i optymalizacji \cite{BarbosaBBKLMMMN22}. 
% Przykłady rzeczywistych zastosowań SMT-solverów w różnych dziedzinach, takich jak weryfikacja oprogramowania, projektowanie układów scalonych czy analiza protokołów komunikacyjnych. na koniec problemy NP-trudne



