\chapter{Teoretyczne podstawy Satisfiability Modulo Theories}

\section{Wprowadzenie do SMT}

\subsection{Problem spełnialności}
Problem spełnialności formuł boolowskich (SAT) jest ważnym problemem algorytmicznym w teorii złożoności obliczeniowej.

Obiektem problemu SAT jest formuła boolowska składająca się jedynie z nazw zmiennych, nawiasów i operacji 
{\wedge} (AND), {\vee} (OR) oraz {\neg} (NOT). Problem polega na tym, czy możliwe jest przypisanie wartości false i true do wszystkich zmiennych występujących w formule, tak aby formuła stała się prawdziwa.

Zgodnie z twierdzeniem Cooka, udowodnionym przez Stephena Cooka w 1971 roku, problem SAT dla formuł boolowskich zapisanych w koniunkcyjnej postaci normalnej jest NP-zupełny. Wymóg zapisu w postaci koniunkcyjnej jest ważny, ponieważ, na przykład, dla formuł reprezentowanych w dysjunkcyjnej postaci normalnej, problem SAT jest trywialnie rozwiązywany w czasie liniowym względem rozmiaru formuły.

\subsection{Satisfiability Modulo Theories}
Satisfiability Modulo Theories (SMT) to dziedzina informatyki teoretycznej, która łączy w sobie problem spełnialności logicznej (SAT) z różnymi teoriami matematycznymi \cite{MouraB11}. 
W kontekście SMT, dany jest zestaw ograniczeń logicznych wyrażonych za pomocą formuł logiki pierwszego rzędu oraz dodatkowe ograniczenia wynikające z konkretnych teorii matematycznych, takich jak teoria liczb całkowitych, teoria równań różniczkowych, czy teoria tablic.
Problematyka SMT polega na stwierdzeniu, czy istnieją wartości zmiennych spełniające zarówno ograniczenia logiczne, jak i dodatkowe ograniczenia wynikające z wybranej teorii matematycznej. W przypadku pozytywnej odpowiedzi, rozwiązaniem problemu jest znaczenie konkretnych wartości zmiennych, które spełniają wszystkie warunki.
	
\section{Historia i rozwój SMT solverów}
Pierwsze próby rozwiązywania problemów SMT miały na celu przekształcenie ich w formuły SAT (na przykład 32-bitowe zmienne zostały zakodowane przez 32 zmienne boolowskie, a odpowiadające im operacje na słowach zostały zakodowane jako niskopoziomowe operacje bitowe) i rozwiązanie formuły za pomocą SAT solvera. Podejście to ma swoje zalety - pozwala na wykorzystanie istniejących solverów SAT bez zmian (As-Is), a także na wykorzystanie wprowadzonych w nich optymalizacji. Z drugiej strony, utrata wysokopoziomowej semantyki leżącej u podstaw teorii oznacza, że solver SAT musi podjąć znaczne działania, aby wywnioskować "oczywiste" fakty (takie jak $x + y = y + x$ dla dodawania). Pomysł ten doprowadził do powstania wyspecjalizowanych solverów SMT, które integrują konwencjonalne dowody boolowskie w stylu algorytmu DPLL z solverami specyficznymi dla teorii ("T solvery"), które działają z dysjunkcjami i koniunkcjami predykatów z danej teorii. 
Satisfiability Modulo Theories(SMT) uogólniają teorię spełnialności boole'owskiej (SAT) poprzez dodanie rozumowania równościowego, arytmetyki, bit-wektorów o stałym rozmiarze, tablic, kwantyfikatorów i innych przydatnych teorii pierwszego rzędu.
SMT solver jest narzędziem do decydowania o spełnialności formuł w tych teoriach. 

\section{Podstawowe teorie logiczne}

\textbf{Teoria Liniowej Arytmetyki Liczb Całkowitych} (Linear Integer Arithmetic) jest fundamentalną gałęzią logiki matematycznej, która zajmuje się manipulacją liczb całkowitych w sposób liniowy, tj. poprzez równania i nierówności liniowe. Problem SMT z teorią tła LIA polega na określeniu spełnialności kombinacji boolowskiej odpowiednich arytmetycznych formuł atomowych i oznaczane jest SMT (LIA). 

SMT (LIA) jest ważne w weryfikacji oprogramowania i zautomatyzowanym wnioskowaniu, ponieważ większość programów używa zmiennych całkowitoliczbowych i wykonuje na nich operacje arytmetyczne. W szczególności SMT (LIA) ma różne zastosowania w automatycznej analizie terminacji, sekwencyjnym sprawdzaniu równoważności i osiągalności stanu w słabych modelach pamięci \cite{CaiLZ22}. Jest to również kluczowa teoria używana przez SMT-solvery do rozwiązywania problemów z zakresu NP-trudnych, takich jak problem plecakowy czy problem pokrycia wierzchołkowego.


\textbf{Teoria Zbiorów} pozwala na manipulację zbiorami elementów oraz wykonywanie operacji takich jak dodawanie, usuwanie, sprawdzanie przynależności, czy operacje na zbiorach (np. przekroje, różnice). Jest szczególnie przydatna w problemach, gdzie dane lub zbiory danych mogą być reprezentowane za pomocą zbiorów, takich jak problem sumy podzbioru czy problem planowania zadań.

\textbf{Teoria Zmiennych Boolowskich (Boolean Variables)}:
Teoria Zmiennych Boolowskich umożliwia operacje na zmiennych logicznych, które mogą przyjąć wartość prawda/fałsz. Pozwala na konstruowanie formuł logicznych, wyrażeń logicznych oraz wykonywanie operacji takich jak koniunkcja, alternatywa, negacja, implikacja czy równoważność. 

\textbf{Teoria Tablic (Arrays)}:
Teoria Tablic umożliwia modelowanie i manipulację strukturami danych, takimi jak tablice. Pozwala na definiowanie tablic, dostęp do ich elementów oraz wykonywanie operacji takich jak przypisanie wartości do elementów tablicy czy odczytywanie wartości z tablicy. Jest przydatna w problemach, gdzie dane są przechowywane w postaci tablic, np. w problemach dotyczących grafów (np. macierzy sąsiedztwa czy list sąsiedztwa), problemach sortowania czy problemach związanych z przetwarzaniem danych.

\section{Zastosowanie SMT w praktyce}
SMT solvery są powszechnie uznawane za niezbędne mechanizmy rozumowania dla różnych obszarów zastosowań, w tym weryfikacja mikroprocesorów z potokiem, sprawdzanie równoważności mikrokodu, testowanie białoskrzynkowe w zastosowaniach związanych z bezpieczeństwem, eksploracja przestrzeni projektowej, a także synteza konfiguracji oraz odkrywanie materiałów kombinatorycznych. \cite{BarbosaBBKLMMMN22}. Ich sukces wynika z kilku czynników, w tym zdolności do pracy z bogatszymi reprezentacjami danych oraz możliwością rozszerzenia pojemności poprzez działanie na poziomie wyższym niż tylko logiczne wartości boolowskie. 

Rozwiązania oparte na SMT znalazły również zastosowanie w obszarach związanych z problemami NP-trudnymi, które są kluczowe dla wielu dziedzin nauki i technologii. Dzięki swojej zdolności do radzenia sobie z złożonymi problemami decyzyjnymi, SMT solvery są wykorzystywane do modelowania i rozwiązywania problemów trudnych do rozwiązania w sposób klasyczny. Przykładowe zastosowania obejmują optymalizację kombinatoryczną, planowanie zasobów, analizę i weryfikację systemów złożonych oraz projektowanie oprogramowania o wysokiej niezawodności. Ponadto, SMT solvery znajdują zastosowanie w dziedzinach badawczych, takich jak sztuczna inteligencja, gdzie są używane do rozwiązywania problemów logicznych i wnioskowania. Ich rosnąca popularność w obszarze problemów NP-trudnych wynika zarówno z ich wydajności, jak i z możliwości pracy z różnymi teoriami i rodzajami ograniczeń, co czyni je wszechstronnym narzędziem w analizie i rozwiązywaniu złożonych problemów decyzyjnych.
