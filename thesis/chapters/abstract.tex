\begin{abstract}
	Celem niniejszej pracy jest analiza i ocena efektywności SMT solverów w rozwiązywaniu klasycznych problemów zaliczanych do klasy NP-trudnych. Przełom w informatyce teoretycznej sprawił, że tego rodzaju zagadnienia stały się przedmiotem intensywnych badań. Niniejsza praca skupia się na zrozumieniu, jak SMT (Satisfiability Modulo Theories) solvery, będące potężnym narzędziem w dziedzinie rozstrzygania logicznego, radzą sobie z tymi wyjątkowo wymagającymi problemami.
	
	Przedmiotem badań w pracy są klasyczne problemy NP-trudne, a celem jest zbadanie i porównanie wydajności trzech SMT solverów - Z3, Yices i cvc5 - w kontekście rozwiązywania konkretnych problemów. Wybór tych narzędzi wynika z ich popularności, wszechstronności i aktywnego udziału w społeczności badawczej.
\end{abstract}