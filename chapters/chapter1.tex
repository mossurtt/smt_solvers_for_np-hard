\chapter{Teoretyczne podstawy SMT i klasy NP-hard}
\section{Złożoność obliczeniowa}

\section{Definicja klasy problemów NP-trudnych}

\section{Spełnialność}


\section{Satisfiability Modulo Theories}
	Satisfiability Modulo Theories (SMT) to dziedzina informatyki teoretycznej, która łączy w sobie problem spełnialności logicznej (SAT) z różnymi teoriami matematycznymi. \cite{MouraB11}
	W kontekście SMT, dany jest zestaw ograniczeń logicznych wyrażonych za pomocą formuł logiki pierwszego rzędu oraz dodatkowe ograniczenia wynikające z konkretnych teorii matematycznych, takich jak teoria liczb całkowitych, teoria równań różniczkowych, czy teoria tablic.
	Problematyka SMT polega na stwierdzeniu, czy istnieją wartości zmiennych spełniające zarówno ograniczenia logiczne, jak i dodatkowe ograniczenia wynikające z wybranej teorii matematycznej. W przypadku pozytywnej odpowiedzi, rozwiązaniem problemu jest znaczenie konkretnych wartości zmiennych, które spełniają wszystkie warunki.
	\subsection{Teorie}
	\subsection{Lazy approach}
	Teoria 
\section{Definicja i zasada działania SMT-solverów}
Teorie spełnialności modulo (SMT) uogólniają teorię spełnialności boole'owskiej (SAT) poprzez dodanie rozumowania równościowego, arytmetyki, bit-wektorów o stałym rozmiarze, tablic, kwantyfikatorów i innych przydatnych teorii pierwszego rzędu.
SMT solver jest narzędziem do decydowania o spełnialności (lub poprawności) formuł w tych teoriach. 
Solvery SMT umożliwiają aplikacje, takie jak rozszerzone sprawdzanie statyczne, abstrakcja predykatów, generowanie przypadków testowych i ograniczone sprawdzanie modelu w nieskończonych dziedzinach.


