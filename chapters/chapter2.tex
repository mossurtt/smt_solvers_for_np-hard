\chapter{Zastosowanie komponentów JavaBeans do tworzenia komponentów interfejsu graficznego}
Rozdział opisuje zastosowanie technologi JavaBeans do tworzenia komponentów
graficznego interfejsu użytkownika ...


\section{Podstawowe informacje dotyczące tworzenia aplikacji internetowych w języku Java}

\subsection{Środowisko potrzebne do działania aplikacji internetowych w Javie}

\subsection{Serwlety}

\subsubsection{Przykładowy serwlet}

W tym fragmencie zaprezentuję jak stworzyć przykładowy serwlet. Nie będzie to jeszcze przykład z użyciem technologii JavaBeans, jednak mam nadzieję, że przybliży on trochę zagadnienie tworzenia serwletów.

\begin{lstlisting}[caption={[Przykładowy serwlet]Przykładowy serwlet}]
import java.io.*;
import javax.servlet.http.*;
public class FirstServlet extends HttpServlet {
	public void doGet(HttpServletRequest request, HttpServletResponse response) throws IOException {
		PrintWriter out = response.getWriter();
		response.setContentType("text/html");
		out.println("Hello <b>World</b> !!!");
	}
}
\end{lstlisting}