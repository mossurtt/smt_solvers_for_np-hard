\chapter{Problemy NP-trudne}

Załóżmy, że jesteśmy menedżerem logistyki, który musi zoptymalizować harmonogram dostaw towarów przez park pojazdów w warunkach ograniczonej sieci dystrybucyjnej obejmującej całe miasto. To pozornie rutynowe zadanie, po bliższym przyjrzeniu się, okazuje się być złożonym problemem optymalizacji kombinatorycznej, charakteryzującym się potrzebą minimalizacji kosztów paliwa, skrócenia czasu podróży i maksymalizacji przepustowości dostaw.

W poszukiwaniu optymalnego rozwiązania mamy do czynienia ze stale rosnącym zestawem zmiennych, w tym nieprzewidywalną dynamiką ruchu, różnymi rozmiarami paczek i dynamicznymi zmianami popytu ze strony klientów. Każda decyzja o ustaleniu konkretnych tras dostawy lub nadaniu priorytetu określonym miejscom docelowym powoduje eksplozję kombinatorycznych możliwości, zmieniając problem optymalizacji logistycznej w przykład problemu NP-trudnego.

Z perspektywy obliczeniowej, problemy NP-trudne są klasą problemów, dla których nie istnieje algorytm wielomianowy, który może zapewnić optymalne rozwiązanie we wszystkich przypadkach. Złożoność ta jest wyraźnie widoczna w problemach takich jak optymalizacja tras, gdzie ogromna przestrzeń rozwiązań nie pozwala na proste rozwiązanie. Złożoność takich problemów logistycznych odzwierciedla szersze trudności występujące w problemach NP-trudnych w różnych obszarach obliczeniowych i optymalizacyjnych.

\section{Złożoność obliczeniowa}

\section{Problem spełnialnośći}

\section{Definicja klasy problemów NP-trudnych}



%\begin{lstlisting}
%\end{lstlisting}