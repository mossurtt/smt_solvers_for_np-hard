\begin{frame}{Problem kolorowania grafu}
\textbf{Definicja}
\begin{itemize}
	\item Problem polega na określeniu minimalnej liczby kolorów potrzebnych do pokolorowania grafu nieskierowanego $G = (V, E)$.
	\item Dwa sąsiednie wierzchołki nie mogą mieć tego samego koloru.	
\end{itemize}
\vspace{10pt}

\textbf{Zastosowania}
\begin{itemize}
	\item Modelowanie problemu kolorowania map za pomocą grafu, w którym każdy wierzchołek reprezentuje kraj, i wierzchołki, których kraje mają wspólną granicę, ze sobą sąsiadują.
	\item Istotny w planowaniu tras, optymalizacji sieci oraz problemach planowania.
\end{itemize}
\end{frame}