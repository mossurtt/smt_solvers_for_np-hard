\begin{frame}{Problem sumy podzbioru: Kodowanie}
Używamy $n$ zmiennych $x_0,…, x_{n−1}$, gdzie $n$ to liczba elementów w $S$.
\vspace{10pt}

\textbf{Warunki dla zmiennych:}
\vspace{5pt}
\begin{itemize}
	\item każda zmienna $x_j$ przyjmuje wartość logiczną $0$ lub $1$.
	\item suma iloczynów $x_i$ i $s_i$ dla wszystkich $k$ elementów ze zbioru $S$ jest równa wartości $t$.
\end{itemize}
\vspace{10pt}

\textbf{Formuła:}
\begin{align*}
	\SubsetSum(n, t) = \left( \bigwedge_{j=0}^{n-1} (x_j = 0 \lor x_j = 1) \right) \land 
	\left( \sum_{i=0}^{n} (x_i * s_i) = t \right)
\end{align*}
\end{frame}
	