\begin{frame}{Ścieżka Hamiltona w grafie skierowanym}
\textbf{Definicja}
\begin{itemize}
	\item Dany graf skierowany $G = (V, E)$, gdzie $V$ to zbiór wierzchołków, a $E$ to zbiór krawędzi.
	\item Zadaniem jest stwierdzenie, czy graf $G$ zawiera ścieżkę Hamiltona.
	\item Ścieżka Hamiltona to sekwencja wierzchołków $s_0, s_1,…, s_{n−1}$ przechodzącą przez każdy wierzchołek dokładnie raz.	
\end{itemize}
\vspace{10pt}

\textbf{Zastosowania}
\begin{itemize}
	\item Istotny w informatyce, telekomunikacji i bioinformatyce.
	\item Kluczowy dla analizy sieci i trasowania w systemach komunikacyjnych.
\end{itemize}
\end{frame}