\begin{frame}{Problem maksymalnego zbioru niezależnego w grafie nieskierowanym}
\textbf{Definicja}
\begin{itemize}
	\item Dany graf nieskierowany $G = (V, E)$, gdzie $V$ to zbiór wierzchołków, $E$ to zbiór krawędzi.
	\item Niezależny zbiór to podzbiór wierzchołków, gdzie żadne dwa nie sąsiadują ze sobą.
	\item Problem maksymalnego zbioru niezależnego polega na znalezieniu największego takiego zbioru w grafie.
\end{itemize}
\vspace{10pt}

\textbf{Zastosowania}
\begin{itemize}
	\item Istotny w teorii grafów i algorytmice.
	\item Ma zastosowanie w wielu problemach optymalizacyjnych i analizie sieci.
\end{itemize}
\end{frame}
	