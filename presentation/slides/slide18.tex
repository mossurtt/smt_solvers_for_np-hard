\begin{frame}{Problem kolorowania grafu: Kodowanie}
Używamy $n$ zmiennych $c_0,…,c_{n−1}$​,gdzie $n$ to liczba wierzchołków w grafie.
\vspace{10pt}

\textbf{Warunki dla zmiennych:}
\vspace{5pt}
\begin{itemize}
	\item Każda zmienna $c_j$ reprezentująca kolor wierzchołka $j$ przyjmuje wartość z przedziału od $1$ do $k$, gdzie $k$ to liczba kolorów.
	\item Dla każdej krawędzi $s,t$ w $E$, kolor wierzchołka $s$ jest różny od koloru wierzchołka $t$.
\end{itemize}
\vspace{10pt}

\textbf{Formuła:}
\begin{align*}
	\GraphColoring(n, E) = \left( \bigwedge_{j=1}^{n} (c_j \geq 1 \land c_j \leq k) \right) \land 
	\left( \bigwedge_{\{s,t\} \in E} (c_s \neq c_t) \right)
\end{align*}	
\end{frame}