\begin{frame}{Problem pokrycia wierzchólkowego}
\textbf{Definicja}
\begin{itemize}
	\item Pokrycie wierzchołkowe grafu nieskierowanego	$G = (V, E)$ to podzbiór $V'\subseteq V$, gdzie każda krawędź ma przynajmniej jeden koniec w $V'$.
	\item Rozmiar pokrycia to liczba wierzchołków w nim zawartych.
	\item Problem polega na znalezieniu minimalnego pokrycia wierzchołkowego w danym grafie.
\end{itemize}
\vspace{10pt}

\textbf{Zastosowania}
\begin{itemize}
	\item Istotny w optymalizacji grafów oraz problemach planowania tras.
	\item Stosowany w analizie sieci, zarządzaniu zasobami oraz problemach pokrycia.
\end{itemize}
\end{frame}