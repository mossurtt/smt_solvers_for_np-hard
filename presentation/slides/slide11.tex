\begin{frame}{Problem maksymalnej kliki w grafie nieskierowanym}
\textbf{Definicja}
\begin{itemize}
	\item Dany graf nieskierowany $G = (V, E)$, gdzie $V$ to zbiór wierzchołków, $E$ to zbiór krawędzi.
	\item Klika to pełny podgraf, gdzie każde dwa wierzchołki są połączone krawędzią.
	\item Maksymalna klika $C \subseteq V$  to taka, która zawiera największą możliwą liczbę wierzchołków.
\end{itemize}
\vspace{10pt}

\textbf{Zastosowania}
\begin{itemize}
	\item Istotny w teorii grafów, sieciach społecznościowych, analizie sieci.
	\item Wykorzystywany w problemach planowania tras, optymalizacji sieci, analizie zależności między obiektami.
\end{itemize}
\end{frame}
	