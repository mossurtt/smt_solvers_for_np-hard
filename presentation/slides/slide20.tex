\begin{frame}{Problem Komiwojażera: Kodowanie}
Używamy $n$ zmiennych $v_0,…, v_{n−1}$, gdzie $n$ to liczba miast.
\vspace{10pt}

\textbf{Warunki dla zmiennych:}
\vspace{5pt}
\begin{itemize}
	\item Właściwą wartość i unikalność.
	\item Istnienie krawędzi między kolejnymi wierzchołkami w trasie oraz zamknięcie trasy.
	\item Ograniczenie sumy wag krawędzi na trasie do wartości $k$.
\end{itemize}

\vspace{10pt}
\textbf{Formuła:}	
\begin{align*}
	&\TSP(n, c, E, k) = \propernumbers(n) \land \distinctvs(n) \land \edges(n, E)  \\
	&\land \left( \bigvee_{\{s,t\} \in E} ((v_{n-1} = s \land v_0 = t) \lor (v_0 = s \land v_{n-1} = t)) \right) \land 
	\left( \bigwedge_{\{s,t\} \in E} \sum c(s,t) \leq k \right)
\end{align*}
	
\end{frame}