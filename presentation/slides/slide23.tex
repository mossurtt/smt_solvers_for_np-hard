\begin{frame}{Generowanie danych wejściowych}
	\begin{itemize}
		\item Do eksperymentów użyto laptopa Dell z procesorem Intel Core i5-1135G7 z częstotliwością 2.40GHz i 16 GB RAMu.
		\item Generowanie danych wejściowych, takich jak grafy, wykonano przy użyciu biblioteki igraph, która zapewnia wszechstronne możliwości manipulacji i analizy grafów.
		
		\item Zdecydowano się generować dwa rodzaje grafów - Barabasi-Alberta i Erdos-Rényi’ego - aby umożliwić badanie efektywności SMT-solverów w różnych warunkach.
		
		\item Do eksperymentów nad problemem SubsetSum wygenerowano zestawy losowych liczb całkowitych w określonym zakresie.
		
	\end{itemize}
\end{frame}
	