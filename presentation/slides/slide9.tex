\begin{frame}{Ścieżka Hamiltona w grafie skierowanym: Kodowanie}
Używamy $n$ zmiennych $v_0,…,v_{n−1}$, gdzie $n$ to liczba wierzchołków.
\vspace{10pt}

\textbf{Formuły uwzględniające warunki dla zmiennych:}
\vspace{5pt}
\begin{itemize}
	\item Zakres zmiennych $v_j$: $\propernumbers(n) = \left( \bigwedge_{j=0}^{n-1} (v_j \geq 0 \land v_j < n) \right) $.
	\item Unikalność zmiennych: $\distinctvs(n) = \left( \bigwedge_{i=0}^{n-1} \bigwedge_{j=i+1}^{n} (v_i \neq v_j) \right)$.
	\item Sprawdzenie krawędzi grafu: $\diredges(n,E) = \left( \bigwedge_{i=0}^{n-1} \bigvee_{(s,t) \in E} (v_i = s \land v_{i+1} = t) \right)$.
\end{itemize}
\vspace{10pt}
\textbf{Cała Formuła:}
\begin{align*}
	\HamPath(n, E) = \propernumbers(n) \land \distinctvs(n) \land \diredges(n, E)
\end{align*}
\end{frame}