\begin{frame}{Problemy NP-trudne}

		 \textbf{Definicja}: Problemy NP-trudne to klasy problemów, które są co najmniej tak trudne jak najtrudniejsze problemy w klasie NP.
		\vspace{10pt}
		
		 \textbf{Klasyfikacja}:
		\begin{itemize}
			\item \textit{Klasa P}: Problemy decyzyjne rozwiązywalne w czasie wielomianowym przez algorytmy deterministyczne.
			\item \textit{Klasa NP}: Problemy, których rozwiązania mogą być zweryfikowane w czasie wielomianowym przez algorytmy niedeterministyczne.
		\end{itemize}
		\vspace{10pt}
		 \textbf{Cechy}:
		\begin{itemize}
			\item Możliwość zredukowania każdego problemu w klasie NP do problemu NP-trudnego w czasie wielomianowym.
			\item Wymagają dużego nakładu obliczeniowego do rozwiązania i weryfikacji, nawet dla relatywnie niewielkich instancji problemów.
		\end{itemize}

\end{frame}