\begin{frame}{Problem Komiwojażera}
\textbf{Definicja}
\begin{itemize}
	\item Problem polega na znalezieniu trasy, która minimalizuje sumaryczny koszt podróży, odwiedzając każde miasto dokładnie raz i kończąc w mieście początkowym.
	\item Dany graf nieskierowany $G = (V, E)$, gdzie wierzchołki reprezentują miasta, a krawędzie odpowiadają możliwym połączeniom między nimi.
	Koszt podróży z miasta $i$ do $j$ jest reprezentowany przez wagę $c(i,j)$.
	\item Celem jest znalezienie trasy o minimalnym koszcie, która odwiedza każde miasto dokładnie raz.
\end{itemize}
\vspace{10pt}

\textbf{Zastosowania}
\begin{itemize}
	\item Logistyka, planowanie tras, optymalizacja sieci komunikacyjnych.
\end{itemize}
\end{frame}