\begin{frame}{Identyfikacja czynników wpływających na efektywność}
	\begin{itemize}
		\item Rozmiar Instancji Problemu:
		Wraz z rosnącym rozmiarem problemu, np. liczbą wierzchołków w grafie lub elementów w zbiorze, czas rozwiązania znacząco wzrasta.
		\item Struktura Grafu:
		Sposób generowania grafu oraz jego złożoność mogą wpływać na wydajność obliczeniową solverów.
		\item Rodzaj Problemu:
		Różnice w wydajności solverów zależą od rodzaju problemu NP-trudnego, np. SubsetSum vs. problemy grafowe.
		\item Zużycie Zasobów:
		Wartość czasowa i pamięciowa solverów różni się w zależności od problemu oraz zastosowanej strategii rozwiązania.
		\item Trudność Spełnialności vs. Niespełnialności:
		Szukanie niespełnialności może być trudniejsze niż spełnialności, co wpływa na czasochłonność rozwiązania problemu.
	\end{itemize}
\end{frame}
	