%%%%%%%%%%%%%%%%%%%%%%%%%%%%%%%%%%%%%%%%%
% Beamer Presentation
% LaTeX Template
% Version 1.0 (10/11/12)
%
% This template has been downloaded from:
% http://www.LaTeXTemplates.com
%
% License:
% CC BY-NC-SA 3.0 (http://creativecommons.org/licenses/by-nc-sa/3.0/)
%
%%%%%%%%%%%%%%%%%%%%%%%%%%%%%%%%%%%%%%%%%

\documentclass{beamer}
\usepackage[utf8]{inputenc}
\usepackage[T1]{fontenc}
\usepackage{polski}

\mode<presentation> {
\usetheme{Frankfurt}
\usecolortheme{dolphin}
}
\usepackage{graphicx}
\usepackage{booktabs}
\usepackage{xcolor}
\usepackage{hyperref}

%------------------------------------------------

\title[Gaming w świecie komputerów]{Gaming w świecie komputerów}

\author{Arthur Zwolski-Valcourt}
\institute[UJD]
{
Uniwersytet\\Humanistyczno-Przyrodniczy\\im. Jana Długosza w Częstochowie \\
\medskip
}
\date{17 stycznia 2019}

%------------------------------------------------

\begin{document}

\begin{frame}
\titlepage
\end{frame}

%------------------------------------------------

% \usebackgroundtemplate{\includegraphics[height=\paperheight]{graphics.jpg}}

%------------------------------------------------

\begin{frame}
\frametitle{Przegląd prezentacji}
\tableofcontents
\end{frame}

%------------------------------------------------

\section{Historia gier}
\subsection{Gry w historii człowieka}

{
\addtobeamertemplate{block begin}{\pgfsetfillopacity{0.85}}{\pgfsetfillopacity{1}}
\usebackgroundtemplate{\includegraphics[width=\paperwidth]{graur2.jpg}}

\begin{frame}
    \begin{block}{Gry w historii człowieka}
    Gry były rozrywką dla ludzi od ponad tysięcy lat i są do dziś popularne. W grach nie ma ograniczeń wiekowych, dla każdego znajdzie się odpowiednia gra.
    \end{block}
\end{frame}
}

\begin{frame}{Pierwsze gry w historii człowieka}
    \textbf{5000 lat p.n.e.}\\
    Kilka tysięcy lat temu ludzie tworzyli już pierwsze gry planszowe
    z~kamienia. Nie miały one nazw, a rozgrywały się wedle nieskomplikowanych zasad - najczęściej wygrywał lepiej zbudowany osobnik. Dziś fakt ten badacze powszechnie uznają za podwaliny całego gatunku bijatyk w grach wideo.
\end{frame}

\begin{frame}{Gra królewska z Ur}
    \begin{columns}[c]
        \column{.6\textwidth}
            \textbf{3000 lat p.n.e.}\\
            To sprzed 5 tys. lat pochodzi plansza
            z~20~polami do gry, pionkami oraz kostkami. Znaleziono ją w miejscowości Ur w Mezopotamii. Gra była prekursorem tryktraka i jest uważana za jedną z najstarszych gier planszowych na świecie.
        \column{.4\textwidth}
        \begin{figure}[h]
            \includegraphics[width=\linewidth]{graur.jpg}
            \caption{\textit{Gra królewska z Ur. Zdjęcie z Muzeum Brytyjskiego (Źródło: wikipedia.pl)}}
        \end{figure}
    \end{columns}
\end{frame}

%------------------------------------------------

\begin{frame}
\frametitle{Gry w pierwszym wieku naszej ery}
    \textbf{I wiek n.e.}\\
    \begin{itemize}
        \item Gry stają się \underline{masowe}. Władcy, pisarze, naukowcy, inteligencja i~ludzie wszystkich stanów grają.\\Niestety, dalszy rozwój cywilizacyjny zmuszał, szczególnie tych ostatnich, także do bardziej \underline{wytężonej}, \underline{regularnej}, często \underline{\textbf{niewolniczej pracy}}. Dostęp do szczytowych, najnowszych produkcji rzymskich czy egipskich jest ograniczany.
        
        \item Rodzi się znane dzisiaj zjawisko \underline{„piractwa”}, drzewiej zwane „koszeniem”. Za skradzioną cezarowi skrzynię z jego ulubioną grą planszową, rzezimieszka stawiano przeciw dwóm tygrysom.
    \end{itemize}
\end{frame}

\begin{frame}
\frametitle{Narodziny współczesnych gier}
    \textbf{1492 rok}\\
    Krzysztof Kolumb dociera do wysp Ameryki Środkowej. Niespełna 500 lat później w USA powstaje pierwsza interaktywna gra - nie planszowa, nie karciana i nie podwórkowa. Jest to gra z użyciem lamp elektronowych - symulator pocisku rakietowego.
    \begin{figure}[h]
        \centering
        \includegraphics[width=0.5\textwidth]{crtamusementdevice.jpg}
        \caption{\textit{,,Cathode-Ray Tube Amusement Device''} (Źródło: \url{newatlas.com})}
    \end{figure}
\end{frame}

%------------------------------------------------

\subsection{Gry w świecie współczesnym}

\begin{frame}
\frametitle{Pionier przemysłu współczesnych gier wideo}
    \begin{columns}[c]
        \column{.6\textwidth}
            \textbf{1922 - narodziny Ralpha Baera,
            Niemcy}\\
            W marcu 1922 roku na świat przychodzi Ralph Baer, stwórca pierwszego prototypu konsoli gier Brown Box (1972). Tuż przed Nocą Kryształową emigruje do Stanów Zjednoczonych. Wynalazca, inżynier i~pionier całego przemysłu gier wideo, które od początku widział jako powszechną, innowacyjną formę rozrywki, dostępną w każdym domu.
        \column{.4\textwidth}
            \begin{figure}
            \includegraphics[width=0.9\linewidth]{RalphBaer.jpg}
            \caption{\textit{W 2006 roku Ralph Baer otrzymał z rąk George'a W. Busha odznaczenie - National Medal of Technology. (Źródło: ralphbaer.com)}}
            \end{figure}
    \end{columns}
\end{frame}

%------------------------------------------------

\begin{frame}
\frametitle{Pierwsza gra elektroniczna (1947)}
    \begin{columns}[c]
        \column{.5\textwidth}
            Zdaniem badaczy gier wideo, pierwszą w pełni elektroniczną grą interaktywną był zaprojektowany w 1947 roku przez dwóch Amerykanów - Thomasa Goldsmitha Jr. i Estle'a Raya - symulator pocisku rakietowego, wykorzystujący układ CRT z ang. Cathode-Ray Tube (później skrótem CRT określano w Polsce monitory kineskopowe).
        \column{.5\textwidth}
            \begin{figure}
                \includegraphics[width=1.0\linewidth]{egun.jpg}
                \caption{\textit{Działo elektronowe jest elementem kineskopów. Odpowiada za wytwarzanie wiązki elektronów o~odpowiedniej energii. (Źródło: wikipedia.pl)}}
            \end{figure}
    \end{columns}
\end{frame}

%------------------------------------------------

\begin{frame}
\frametitle{Rzwój technologii i gier}
    Wraz z postępem technologii, gry jak i również programy rozwijały się wykładniczo przez lata 1970-2012. Stagnacja technologii jest widoczna w ostatnich latach, gdzie rozwój już się nie skupia na prędkości rdzenia lecz na ilości rzeczywistych rdzeni.
    \begin{figure}
    \includegraphics[width=.65\textwidth]{moore.png}
                \caption{\small{Stagnacja prawa Moore'a (Źródło: techspecs.blog)}}
            \end{figure}
\end{frame}

%------------------------------------------------
\section{Gaming we współczesnym świecie}
\subsection{Różne platformy gier}
%------------------------------------------------

{
\addtobeamertemplate{block begin}{\pgfsetfillopacity{0.85}}{\pgfsetfillopacity{1}}
\usebackgroundtemplate{\includegraphics[height=\paperheight]{horizon.jpg}}
\begin{frame}
    \begin{block}{Nowoczesne gry}
    Gry widziały największy postęp w ostatnich latach. Są one bardziej zaawansowane niż kiedykolwiek, często warte miliony w dochodach oraz niektóre, najbardziej popularne z nich posiadają swój własny tytuł w E-Sports.
    \end{block}
\end{frame}
}

%------------------------------------------------

\begin{frame}
\frametitle{Gry na konsolę}
    Obecnie bez wątpienia komputer wygrywa na rynku gier. Większość gier jest kompatybilna z systemem operacyjnym Windows, lecz są też i gry na konsole takie jak Play Station 4 czy Xbox One.
    \break
    \begin{columns}
        \column{.5\textwidth}
            \begin{figure}
                \includegraphics[width=.5\textwidth]{ps4.png}
                \caption{Konsola Play Station 4 (Źródło: playstation.com)}
            \end{figure}
        \column{.5\textwidth}
            \begin{figure}
                \includegraphics[width=.8\textwidth]{xboxone.png}
                \caption{Konsola Xbox One X (Źródło: xbox.com)}
            \end{figure}
    \end{columns}
\end{frame}

%------------------------------------------------

\begin{frame}
\frametitle{Gry mobilne}
    Istnieją również, oprócz gier komputerowych i na konsolę, gry mobilne. Są one równie popularne, gdyż występują na wszystkich smartfonach i kompatybilnych urządzeniach mobilnych.
    \break
    \begin{columns}
        \column{.5\textwidth}
            \begin{figure}
                \includegraphics[width=0.8\textwidth]{tetris.png}
                \caption{Pierwsza gra mobilna, Tetris (Źródło: wikipedia.com)}
            \end{figure}
        \column{.5\textwidth}
            \begin{figure}
                \includegraphics[width=.8\textwidth]{sonic.jpg}
                \caption{Nowoczesna implementacja gry mobilej Sonic Hedgehog (Źródło: greenbot.com)}
            \end{figure}
    \end{columns}
\end{frame}

%------------------------------------------------

\begin{frame}{Gry komputerowe}
    Wszystkie gry czy to na konsolę, na telefon czy na komputer, są najbardziej uzależnione od kart graficznych. W 1999 roku pierwsza karta graficzna została wyprodukowana przez NVidię. AMD, który był i do dziś jest największą konkurencją dla NVidii na polu kart graficznych również wyprodukował swoją pierwszą kartę graficzną kompatybilną z DirectX 7 w 2000 roku.
    \break
    \begin{columns}
        \column{.5\textwidth}
            \begin{figure}
               \small \includegraphics[width=0.6\textwidth]{firstgpu.jpg}
                \caption{Pierwsza karta graficzna NVidii, GeForce 256, rok 1999 (Źródło: wikipedia.pl)}
            \end{figure}
        \column{.5\textwidth}
            \begin{figure}
               \small \includegraphics[width=.6\textwidth]{rv100.jpg}
                \caption{Karta graficzna AMD, RV100, rok 2000 (Źródło: wikipedia.pl)}
            \end{figure}
    \end{columns}
\end{frame}

%------------------------------------------------

\subsection{Sprzęt gamingowy}

\begin{frame}
\frametitle{Sprzęt do gier}
    W dzisiejszych czasach sprzęt komputerowy jest o tyle zaawansowany, że nie musimy kompresować naszych 100-200 linijkowych gier aby się zmieściły na 40 Megabajtowym "kartridżu". Najnowsze gry osiągają już ponad 100 GB miejsca na dysku. Tak więc omówimy teraz, jak najlepiej zbudować komputer na dzisiejszych standardach.
\end{frame}

%------------------------------------------------

\begin{frame}
\frametitle{Komponenty}
    Rozbijmy komputer na komponenty i omówmy jak zbudować komputer:
    \begin{itemize}
        \item Karta graficzna
        \item Procesor
        \item Pamięć podręczna (RAM)
        \item Pamięć dysku (HDD/SSD/SSHD)
    \end{itemize}
\end{frame}

%------------------------------------------------

\begin{frame}
\frametitle{Karta graficzna do gier}
    Zaczynając od najważniejszego komponentu jakim jest karta graficzna. Wszystko zależy od tego co chce się zbudować. Jednak do gier, liczy się zarówno monitor do którego jest połączona karta graficzna oraz jakie Frames Per Second (FPS) czyli klatki na sekundę chcemy osiągnąć.\\
    \underline{Najczęściej liczy się jak największy licznik FPS!}
\end{frame}

%------------------------------------------------

\begin{frame}
\frametitle{Procesor do gier}
    W przypadku procesora, to już zależy jak gra wykorzystuje rdzenie procesora. Czy wykorzystuje wszystkie rdzenie? Czy poprawnie zakańcza thready?\\
    Najczęściej procesor nie jest tak ważny jak karta graficzna, ale ma on dużo większe znaczenie przy \underline{renderowaniu filmów czy animacji}.
\end{frame}

%------------------------------------------------

\begin{frame}
\frametitle{Pamięć RAM}
    Pamięć ram również ma duże znaczenie w grach, bo to od niej zależy \underline{jak szybko procesor może pracować na danych w nich zapisanych}.\\
    Najważniejsze parametry na które musimy zwrócić uwagę przy kupnie RAM'u, jest ich \textbf{maksymalna prędkość (MHz)} oraz \textbf{ich opóźnienie (CLXX)}. Im mniejsze opóźnienie, tym lepszy jest RAM.\\
    \underline{Opóźnienie może się zmniejszyć} jeśli kupimy karty RAM o wysokich prędkościach i je spowolnimy do mniejszych.
\end{frame}

%------------------------------------------------

\begin{frame}
\frametitle{Pamięć dysku}
    Pamięć dysku jest niczym miejsce pracy w którym możemy zapisać wszelakie pliki i na nich pracować nawet po restarcie systemu (kiedyś wpisywano gry po 6 godzin aby w nie grać przez 15 minut, a po restarcie komputera były one stracone i trzebabyło wpisywać je na nowo).\\
    Najszybsze (a za razem najdroższe) dyski są dyski SSD (Solid State Disk). Przy kupnie dysków SSD najważniejsze jest czy jest ono SLC, MLC, TLC czy QLC. Najlepsze, najszybsze i z największym cycklem zapisywania są SSD z \underline{\textbf{SLC}}. Najtańsze i najwolniejsze są z QLC. Proszę pamiętać, iż VNAND jest MLC!
\end{frame}

%------------------------------------------------

\section{Zakończenie}
\subsection{Podsumowanie}

\begin{frame}
\frametitle{Podsumowanie}
    Moja rada, na najlepszą kombinację komonentów podstawowych przy wyborze komputera do gier:
    \begin{itemize}
        \item Karta graficzna posiadająca najlepszą wartość pieniądza do ilości klatek (często są to te ze średniej półki jak GTX 1070 lub RX 570 w przypadku AMD)
        \item Ekran FullHD 1920x1080, odświeżanie FreeSync bądź GSync 144Hz, 1ms spóźnienia
        \item Procesor i7 w przypadku intela bądź Ryzen 5 w przypadku AMD
        \item Pamięć ram 2400+ Mhz, jak najmniejszy CL (CL14 lub niżej)
        \item Podwójny układ SSD SLC po 60GB w Raid0, i dodatkowo 1TB lub więcej HDD 7200RPM
    \end{itemize}
\end{frame}

%------------------------------------------------

\begin{frame}
\Huge{\centerline{Dziękuję za uwagę!}}
\end{frame}

%------------------------------------------------

\subsection{Bibliografia}

\begin{frame}[allowframebreaks]
\frametitle{Bibliografia}

\end{frame}

%------------------------------------------------

\end{document}