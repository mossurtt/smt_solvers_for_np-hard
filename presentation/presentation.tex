%%%%%%%%%%%%%%%%%%%%%%%%%%%%%%%%%%%%%%%%%
% Beamer Presentation
% LaTeX Template
% Version 1.0 (10/11/12)
%
% This template has been downloaded from:
% http://www.LaTeXTemplates.com
%
% License:
% CC BY-NC-SA 3.0 (http://creativecommons.org/licenses/by-nc-sa/3.0/)
%
%%%%%%%%%%%%%%%%%%%%%%%%%%%%%%%%%%%%%%%%%

\documentclass{beamer}
\usepackage[utf8]{inputenc}
\usepackage[T1]{fontenc}
\usepackage{polski}
\usepackage[polish]{babel}
\mode<presentation> {
	\usetheme{Madrid}
	\usecolortheme{crane}
}
\usepackage{graphicx}
\usepackage{booktabs}
\usepackage{xcolor}
\usepackage{hyperref}

%------------------------------------------------

\title{Efektywność SMT solverów dla klasycznych problemów NP-trudnych}

\author{Tetiana Mossur}
\institute[UJD]
{
	Uniwersytet Jana Długosza w Częstochowie \\
	\medskip
}
\date{}

%------------------------------------------------

\begin{document}
	
	\begin{frame}
		\titlepage
	\end{frame}

	%------------------------------------------------

	\begin{frame}{Sformułowanie problemu}
		SMT (Satisfiability Modulo Theories) stanowi aktywny obszar badań w informatyce, umożliwiający efektywne rozwiązywanie problemów z użyciem metod wnioskowania logicznego. Dzięki postępowi  w technologii SMT, zaawansowane solvery znajdują zastosowanie  w wielu dziedzinach, od weryfikacji procesorów po analizę statyczną. 
		Niniejsza praca ma na celu zbadanie i ocenę efektywności trzech popularnych SMT solverów w języku Python - Z3, Yices i CVC5 –  w rozwiązywaniu klasycznych problemów NP-trudnych. Analiza uzyskanych wyników pozwoli ocenić efektywność każdego solvera,  co stanowi istotny wkład w zrozumienie ich praktycznego zastosowania  w różnych obszarach.
	\end{frame}
	
	%------------------------------------------------	
	
	\begin{frame}
		\Huge{\centerline{Dziękuję za uwagę!}}
	\end{frame}
	
	%------------------------------------------------
	

	%------------------------------------------------
	
\end{document}