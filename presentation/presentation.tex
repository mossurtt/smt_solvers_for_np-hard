%%%%%%%%%%%%%%%%%%%%%%%%%%%%%%%%%%%%%%%%%
% Beamer Presentation
% LaTeX Template
% Version 1.0 (10/11/12)
%
% This template has been downloaded from:
% http://www.LaTeXTemplates.com
%
% License:
% CC BY-NC-SA 3.0 (http://creativecommons.org/licenses/by-nc-sa/3.0/)
%
%%%%%%%%%%%%%%%%%%%%%%%%%%%%%%%%%%%%%%%%%

\documentclass{beamer}
\usepackage[utf8]{inputenc}
\usepackage[T1]{fontenc}
\usepackage{polski}
\usepackage[polish]{babel}
\mode<presentation> {
	\usetheme{Madrid}
	\usecolortheme{crane}
}
\usepackage{graphicx}
\usepackage{booktabs}
\usepackage{xcolor}
\usepackage{hyperref}

%------------------------------------------------

\title{Efektywność SMT solverów dla klasycznych problemów NP-trudnych}

\author{Tetiana Mossur}
\institute[UJD]
{
	Uniwersytet Jana Długosza w Częstochowie \\
	\medskip
}
\date{}

%------------------------------------------------

\begin{document}
	
	\begin{frame}
		\titlepage
	\end{frame}

	%------------------------------------------------

	\begin{frame}{Sformułowanie problemu}
		Niniejsza praca ma na celu zbadanie i ocenę efektywności trzech popularnych SMT solverów - Z3, Yices i CVC5 –  w rozwiązywaniu klasycznych problemów NP-trudnych. Analiza uzyskanych wyników pozwoli ocenić efektywność każdego solvera, co stanowi istotny wkład w zrozumienie ich praktycznego zastosowania w różnych obszarach.
	\end{frame}

	%------------------------------------------------	
		
	\begin{frame}{Zastosowanie SMT solverów}
		
	\end{frame}
	
	%------------------------------------------------
	
	\begin{frame}{Rozdział 1: Teoretyczne podstawy Satisfiability Modulo Theories}
		W pierwszym rozdziale zawarłam teoretyczne fundamenty Satisfiability Modulo Theories (SMT), wprowadzając czytelnika  w główne aspekty tej dziedziny. Omówiłam kluczowe pojęcia, biorąc pod uwagę rozwój SMT-solverów i ich praktyczne zastosowania w realnych sytuacjach, takich jak te związane z klasą problemów NP-trudnych.
	\end{frame}
	
	%------------------------------------------------	
	
	\begin{frame}{Rozdział 2: Problemy NP-trudne}
		Drugi rozdział przedstawia podstawowe pojęcia złożoności obliczeniowej oraz koncepcji spełnialności, dwóch kluczowych aspektów niezbędnych do zrozumienia, czym są problemy obliczeniowe w kontekście SMT. (%Następnie dokonywana jest definicja)% 
		klasy problemów NP-trudnych, które stanowią centralny temat niniejszej pracy magisterskiej.
	\end{frame}
	
	%------------------------------------------------
	
	\begin{frame}{Rozdział 3: Przegląd SMT Solverów: Z3, Yices, CVC5}
		W rozdziale trzecim zawarłam szczegółowy przegląd trzech popularnych SMT-solverów: 
		Z3, 
		Yices i 
		CVC5, 
		podkreślając ich cechy, mocne strony i potencjalne ograniczenia. 
		Czytelnik zdobywa wgląd w różnice między tymi narzędziami, co stanowi podstawę dla późniejszych badań.
	\end{frame}
	
	%------------------------------------------------
	
	\begin{frame}{Rozdział 4: Kodowanie problemów}
		Rozdział czwarty skupia się na przedstawieniu 12 klasycznych problemów NP-trudnych, a następnie opisuje sposób ich kodowania w języku Python.
	\end{frame}
	
	%------------------------------------------------
	
	\begin{frame}{Rodzdiał 5: Eksperymenty i analiza wyników}
		Rozdział piąty poświęcony jest praktycznym eksperymentom, wykorzystując wyżej przedstawione solvery do rozwiązania wybranych problemów obliczeniowych. Analiza wyników pozwoli określić efektywność każdego z nich i wyciągnąc wnioski co do ich zastosowania.
	\end{frame}
	
	%------------------------------------------------
	
	\begin{frame}{Przegląd literatury}
		SMT jest dynamiczną i bardzo rozwijającą się dziedziną badań o wielu praktycznych zastosowaniach. 
		Kluczową rolę odgrywają prace naukowe autorstwa Leonardo de Moura  i Nikolaj Bjørner. Ich wkład w rozwój SMT obejmuje różnorodne aspekty,  od podstaw teoretycznych po praktyczne zastosowania.
		Moura i Bjørner byli współtwórcami SMT solvera Z3, który jest szeroko stosowany, w tym w mojej pracy magisterskiej. 
		Ich praca 'Satisfiability Modulo Theories: An Appetizer' zawiera krótki przegląd SMT i głównych koncepcji technicznych: wprowadza do logiki proposycyjnej, omawiając podstawowe pojęcia, takie jak: formuły proposycyjne, spełnialność, walidność i równoważność; przechodzi również przez konwersję formuł proposycyjnych do postaci koniunktowej normalnej (CNF); przedstawia efektywne techniki analizy przypadków oraz pojęcie teorii, prezentując różne teorie integrowane z rozwiązywaczami SMT, takie jak arytmetyka liniowa, różnicowa, nieliniowa, arytmetyka bitów, teoria tablic, itp.
	\end{frame}
	
	%------------------------------------------------
	
	\begin{frame}{Przegląd literatury}
		Kolejne autorzy, Clark Barrett i Cesare Tinelli, są wiodącymi badaczami  w dziedzinie weryfikacji formalnej. Znani ze swojego znaczącego wkładu w SMT, rozwinęli integrację rozumowania teorii w narzędziach do automatycznej weryfikacji. Ich wspólna praca przyczyniła się do ukształtowania podstaw i praktycznych zastosowań SMT w obszarach takich jak sprawdzanie modeli i weryfikacja oprogramowania.
		Rozdział 11 z książki 'Handbook of Model Checking' - praca Barretta i Tinelliego - zawiera kompleksowy przegląd SMT z naciskiem na "leniwe podejście", powszechną metodę implementacji solverów SMT. Rozdział ten obejmuje podstawy techniczne, wprowadza współpracę między SAT i solverami teorii oraz bada solvery teorii  dla różnych teorii tła. Omówiono także techniki łączenia solverów teorii  i przedstawiono rozszerzenia podejścia leniwego. Wyjaśniono również alternatywne podejście, "eager approach", wykorzystujące solwery SAT w bardziej bezpośredni sposób, a ponadto, podkreśla kluczowe funkcje nowoczesnych solverów SMT wykraczające poza sprawdzanie satysfakcji, pokazując ich znaczenie  w zastosowaniach takich jak sprawdzanie modeli.
	\end{frame}
	
	%------------------------------------------------
	
	\begin{frame}{Przegląd literatury}
		Dennis Yurichev, nastepny autor w tym przeglądzie literatury, jest znaczącą postacią w dziedzinie walidacji oprogramowania. Jego praca, "SAT/SMT by Example", służy jako praktyczny przewodnik po implementacji solverów SAT i SMT. Dzięki swojemu doświadczeniu w tej dziedzinie, Yurichev dostarcza cennych spostrzeżeń poprzez ilustrujące przykłady, dzięki czemu książka jest niezbędnym źródłem informacji zarówno dla początkujących, jak  i doświadczonych specjalistów zainteresowanych praktycznymi aplikacjami  i implementacjami technik rozwiązywania SAT i SMT.
	\end{frame}
	
	%------------------------------------------------
	
	\begin{frame}
		\Huge{\centerline{Dziękuję za uwagę!}}
	\end{frame}

	%------------------------------------------------
	
\end{document}